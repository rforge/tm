\documentclass[a4paper]{article}
\usepackage{url,a4wide}

\newcommand{\file}[1]{`\textsf{#1}'}
\newcommand{\strong}[1]{{\normalfont\fontseries{b}\selectfont #1}}
\let\pkg=\strong

\title{Text Mining and PDF}
\author{Kurt Hornik}

\begin{document}

\maketitle{}

A considerable amount of text of possible interest for text mining is
nowadays contained in Portable Document Format (PDF) files.  In fact, we
have been told that the best way to extract the text from a \LaTeX{}
document is to create PDF output from the document and extract the text
from the PDF file.  Thus, package \pkg{tm} provides the \verb|readPDF()|
reader function for PDF files, which extracs both the text and the meta
data in these files.

The current implementation is based on programs \verb|pdfinfo| and
\verb|pdftotext|, which on Debian-based systems are in package
\verb|poppler-utils|, which provides several command line utilities
(based on the Poppler library) for working with PDF files, and replaces
the former \verb|xpdf-utils|.  However, apparently these utilities are
not necessarily easily available on other systems (Windows?).  In
addition, it might be more appropriate to work more directly with the
Poppler library, or employ different libraries for handling PDF files.
This document discusses available alternatives.

Two years ago we added code for reading information in PDF files to the
base R \pkg{tools} package, partially borrowing ideas from PDFMiner, a
Python module for extracting information from PDF documents which
``focuses entirely on getting and analyzing text data'' (Debian package
\verb|python-pdfminer|).  For some time, the metadata extraction in the
\pkg{tm} PDF reader used \verb|tools:::pdf_info()|.  However, it turns
out that considerable additional effort would be needed to make even the
basic information extraction fully functional (e.g., adding support for
encryption), which would nevertheless yield only a rather limited set of
PDF tools: so clearly, this route must be abandoned, and the code
eventually be removed again.  (Which is somewhat unfortunate, as it
provides a rather nice low-level interface to the objects in the PDF
files.)

We have also written a package \pkg{Rpoppler} which directly interfaces
the Poppler library (specifically, using the Glib interface).  This
works rather well, and is certainly more efficient than employing the
shell utilities which use the Poppler library.  However, the portability
issues will remain (installation needs the Poppler Glib interface
development files: on Debian, they are conveniently available in package
\verb|libpoppler-glib-dev|, on Fedora in \verb|poppler-glib-devel|),
whereas the situation on other systems may be more complicated, and in
particular we have not yet been able to determine whether the package
can be made available for Windows).  The package was published on CRAN
on 2013-08-08, with OS type unix for the time being.

Web search reveals several pointers to alternative functionality, e.g.,
\url{http://stackoverflow.com/questions/6187250/pdf-text-extraction} and 
\url{http://stackoverflow.com/questions/3650957/how-to-extract-text-from-a-pdf}.
We ignore commercial solutions.  The Python PDFminer module was already
mentioned.  Calibre and Abiword can conveniently perform text
extraction, but as far as I can see themselves use Poppler, with results
identical to those of \verb|pdftotext|, and it seems there is no way to
extract the metadata using either program.

Ghostscript provides functionality for PostScript/PDF preview and
printing.  The Ghostscript interpreter is employed in package
\pkg{tools} for PDF compaction, and seems commonly available on Linux
(not sure about Mac OSX), with Windows binaries available from
\url{http://www.ghostscript.com/download/gsdnld.html}.  Ghostscript can
extract text from PDF files either directly via the txtwrite output
device, or via first converting to Postscript (e.g., using program
\verb|pdf2ps|) and then extracting the text (e.g., using program
\verb|ps2txt|).  Both programs in fact call the Ghostscript interpreter
(the first using the ps2write output device, the second additionally
using the \file{ps2ascii.ps} program that comes with every Ghostscript
installation), so that alternatively one can directly call the
Ghostscript interpreter to extract the text.  It is also possible to
extract information using Ghostscript (e.g.,
\url{http://stackoverflow.com/questions/2943281/using-ghostscript-to-get-page-size}).
This needs file \file{pdf\_info.ps} available in the \file{toolbin}
directory of the Ghostscript sources.  This extracts information
somewhat differently from \verb|pdfinfo|, but in a form which should be
conveniently usable for a PDF reader in package \pkg{tm}.  However, PDF
strings in UTF-16BE are not encoded correctly, and PDF dates are left in
PDF date format: the latter could be handled via
\verb|tools:::PDF_Date_to_POSIXt()| (as long as this is available).

The PoDoFo library (\url{http://podofo.sourceforge.net}) is a ``free,
portable C++ library'' which includes classes to parse PDF files and
extract information from these, and claims to be well-documented (see
\url{http://podofo.sourceforge.net/support.html#api}).  Similar to
Poppler, there is also a collection of shell tools for working with PDF
files using the PoDoFo library, see
\url{http://podofo.sourceforge.net/tools.html}.  On Debian, these tools
are in package \verb|libpodofo-utils|, which includes
\verb|podofopdfinfo| and \verb|podofotxtextract| for extracting
information and text.  Assuming the tools are useful, one might also
explore the potential of writing an R interface package for the library
(in particular, if this also allows for more low-level access to the PDF
objects than the Poppler Glib interface does).  However: currently,
the library does not provide access to the CreationDate and ModDate
information from the Info dictionary, and text extraction seems rather
awful.

Finally, the Perl CAM::PDF PDF manipulation library
(\url{https://metacpan.org/release/CAM-PDF}, not mentioned in the above
posts) provides a variety of Perl script utilities, including
\file{pdfinfo.pl} and \file{getpdftext.pl} for extracting information
and text, respectively.  On Debian, the library is available as package
\verb|libcam-pdf-perl|, with these scripts made available as programs
\verb|pdfinfo.cam-pdf| and \verb|getpdftext|.  Assuming Perl can be made
available on all platforms supported by R, it seems one could create an
R package providing the CAM::PDF modules and its dependencies (the
Debian package depends on \verb|libcrypt-rc4-perl| and
\verb|libtext-pdf-perl|) and the two \file{.pl} scripts using these
modules.  The output from CAM::PDF pdfinfo is very similar to that of
poppler-utils pdfinfo (the dates still seem to come in PDF format,
though).  I have not checked (yet) how CAM::PDF text extraction compares
to Poppler text extraction.  However, it seems that CAM::PDF has
problems handling PDF strings in UTF-16BE (e.g., run its pdfinfo on file
\file{AlleleRetain\_User\_Guide.pdf} in package \pkg{AlleleRetain}).

My conclusions would be as follows.
\begin{enumerate}
 \item There are several ``reasonable'' mechanisms for extracting
  information and text from PDF files for possible use in the PDF reader
  in package \pkg{tm}.  We should thus provide a mechanism to select the
  desired ``engine'', both as an argument to the reader generator and as
  an option for package \pkg{tm} (there currently seems no mechanim for
  package-level options, comparable to \verb|sets::sets_options()|).

 \item Currently \verb|tm::readPDF()| hard-wires using \verb|pdftotext|:
  this would need changing.

 \item Every engine needs to provide a function for extracting
  information to be used for the document metadata, and a function for
  extracting the text to be used for the document content.  For the
  former, we need to be able to extract the Author, CreationDate,
  Subject, Title and Creator, ideally via \verb|$| subscripting with the
  above name literals.  The dates would need to be POSIXt objects.

 \item We now have Rpoppler and Ghostscript (`gs') engines, so it makes
  perfect sense to start working on the engine mechanism.  For both
  engines, integration into package \pkg{tm} is trivial by using
  \verb|Rpoppler::PDF_info()| and \verb|Rpoppler::PDF_text()|, and
  \verb|pdf_info_via_gs()| and \verb|pdf_text_via_gs()|, respectively.

 \item The Ghostscript engine seems most portable, but is rather slow,
  and there are a few PDF files Ghostscript cannot read.  So it should
  be the fallback engine, to be used when no others were configured or
  found to be available.
\end{enumerate}

\end{document}
